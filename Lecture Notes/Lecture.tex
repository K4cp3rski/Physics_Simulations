\documentclass[12pt,a4paper]{report}
\usepackage[utf8]{inputenc}
\usepackage[T1]{fontenc}
\usepackage{fontspec}
\usepackage[polish]{babel}
\usepackage{amsmath}
\usepackage{graphicx}
\usepackage[table,xcdraw]{xcolor}
\usepackage{hhline}
\usepackage{placeins}
\usepackage[margin=0.6in]{geometry}
\usepackage{appendix}
\usepackage{colortbl}
\usepackage{physics}
\usepackage{float}
\usepackage{datetime}
\usepackage{makeidx}
\usepackage{hyperref}
\usepackage{verbatim}

\title{Computer Simulations in Physics 2021/2022}
\author{Kacper Cybiński}
% \newdate{date}{28}{01}{2022}
% \date{\displaydate{date}}
\date{\today}
\makeindex
\setlength\parindent{0pt}

\addto\captionspolish{\renewcommand{\chaptername}{Lecture}}

\newcommand{\ind}[1]{{\color{blue} #1\index{#1}}}

\newcommand{\com}[1]{{\color{red} #1}}

\newcommand{\link}[2]{{\color{cyan} \href{#1}{#2}}}

\newcounter{lec}

\newenvironment{lecture}[1]{\par\medskip
   \noindent\chapter{#1} \rmfamily}{\medskip}

\renewcommand{\emph}{\textbf}

\begin{document}
\maketitle
\thispagestyle{empty}


\begin{lecture}{Python Introduction}
\textit{Python Introduction, very general}.
\href{https://www.fuw.edu.pl/~qba/cmpp2022}{Course website}\\
\href{http://studenci.fuw.edu.pl/~kc427902/CMCS_2022/lect1.pdf}{Lecture 1 presentation}\\
Packages needed for the course:
\begin{itemize}
    \item NumPy
    \item SciPy
    \item MatPlotLib
\end{itemize}
Tools needed:
\begin{itemize}
    \item Python $>3.0$
    \item Jupyter Notebook \textit{(optional)}
\end{itemize}
Interesting submodules of \emph{scipy}:
\begin{itemize}
    \item \verb|scipy.constants| - Physical constants
    \item \verb|scipy.special| - Special Functions
    \item \verb|scipy.integrate| - \verb|scipy.integrate.quad|  is an interface for \verb|QUADPACK| FORTRAN integration package
\end{itemize}
The presentations for tutorials: \href{http://studenci.fuw.edu.pl/~kc427902/CMCS_2022/lab1a.pdf}{Basic Python Tutorial}
\href{http://studenci.fuw.edu.pl/~kc427902/CMCS_2022/lab1b.pdf}{Intro to NumPy and MatPlotLib}
\emph{Cheatsheets}:
\begin{itemize}
    \item \link{http://studenci.fuw.edu.pl/~kc427902/CMCS_2022/python_cs.pdf}{Python Cheatsheet}
    \item \link{http://studenci.fuw.edu.pl/~kc427902/CMCS_2022/numpy_cs.pdf}{NumPy Cheatsheet}
    \item \link{http://studenci.fuw.edu.pl/~kc427902/CMCS_2022/matplot_cs.pdf}{Matplotlib Cheatsheet}
\end{itemize}
\end{lecture}

\begin{lecture}{Complexity: Crackling noise, avalanches, and hysteresis}
\section{Grading}
\begin{itemize}
    \item lecture and lab combined
    \item one lab - 1.0 points for all excercises finished on the spot
    \item bring the excercises next week - 0.8 points
    \item Bonus, extra excercises (take-home), 0.2 points each
    \item Pass - 50\% points
    \item Presentation (5 min, obligatory) - i.e. presentation of last week results. - 1.0 point
    \item Last week ($8-9.06$) final presentation time slot, and written exam (8 questions -  4 Daniel, 4 Tworzydło) - $8 \cdot 0.4 = 3.2$ pts, which accounts for 20\% of course points, 50\% needed for pass.
\end{itemize}

\section{Crackling}
\begin{itemize}
    \item The system responds through discrete interactions,
    \item Events span a large range of sizes,
    \item We only look at macroscopic effects, not the microscopic.
\end{itemize}
Crackling systems examples:
\begin{itemize}
    \item Fireplace
    \item Earthquakes
    \item Crampled Paper
    \item Magnetic material in external field
\end{itemize}

\subsection{Earthquakes - \ind{\link{https://en.wikipedia.org/wiki/Gutenberg–Richter_law}{Gutenberg-Richter Law}}}
The relation frequency versus magnitude:
\[
        N \propto 10^{-\alpha M} \propto E^{- \frac{2 \alpha}{3}}
\]
It's a power law, and they are realted to scale invariance.

\section{Avalanches}

Avalanches happen because the systems naturally end up at the critical point. "self organised criticality" (\link{https://en.wikipedia.org/wiki/Self-organized_criticality}{SOC}). It was formulated by Bak.\\

\emph{Crackling studies in physics}, i.e. : Bubbles in foams, fluids in porous materials, fractures of discordered materials, fluctuations in stock markets, cascading failures in power grids.

\subsection{Magnet with random fields}

\link{https://sethna.lassp.cornell.edu/SimScience/crackling/Advanced/Magnets/BarkhausenExperiment.html}{Barkhausen noise experiment} - magnetic domains flipping in external $H(t)$, which is turned into electric signal, and audible through the speaker.\\

What we will use here is a \emph{Random field \ind{\link{https://en.wikipedia.org/wiki/Ising_model}{Ising Model}}}. Its energy function is:
\[
    \mathcal{H} = - J \sum_{\ev{i, j}} s_i s_j - \sum_i (H(t) + h_j)s_i
\]
Its properties:
\begin{itemize}
    \item local, Gaussian distributed $h_j$ with std deviation $\sigma^2 = R$
    \item As spins initially point down ($s_j = -1$)
    \item The magnetic field $H(t)$ is slowly increasing, slowly enough for the system to reach the ground state
    \item The spins only flip to decrease energy
    \item It is all located on a 2D grid, with spind being the knots of the net
    \item Magnetization lags behind the field - we get histeresis, formulated as a series of small, sharp jumps
\end{itemize}
\[
    M = \frac{1}{c^2} \sum_i s_i
\]

\section{Lab Introduction}
Process:
\begin{itemize}
    \item Each spin flips when it can get more energy
    \item Local field $J \sum_{j(i)} s_j\footnote{neighbours of spin} + h_i + H(t)$ at site $i = (\cdot, \cdot)$ changes sign
    \item Spin change can be trigerred by
    \begin{itemize}
        \item one of the neighboring flips
        \item Increase of $H(t)$, to $H(t) = 0$
    \end{itemize}
\end{itemize}

\link{http://studenci.fuw.edu.pl/~kc427902/CMCS_2022/lect2.pdf}{Lecture Slides}
\link{http://studenci.fuw.edu.pl/~kc427902/CMCS_2022/lab2.pdf}{Labs Introduction}

\end{lecture}

% -----------------------------------------------------------------------
% Wykład 17.03.2022

\begin{lecture}{Computational Complexity once again}
\section{Complexity Classes}
Theory of computation - study how resources scale with size $N$ of the problem. IRL - what's the behavour of the least efficient part of algorithm when size goes to $N$\\
\com{Paste graphic from Tworzydło slides}
Problems with exponential time ($t \sim 2^N$) are called "looking for a needle in a haystack"
\subsection{Boolean computing - SAT problems}
Here we have two terms which will appear in this
\begin{itemize}
    \item CNF (\link{https://en.wikipedia.org/wiki/Conjunctive_normal_form}{Conjunctive normal form}) 
    \item kSAT - SAT problem with at most $k$ variables
\end{itemize}
\emph{Profit?} We can reduce an exponential problem to a polynomial (here: quadratic) time problem

\subsection{Spin Glass}
The problem leading to spin glass mapping being a NP-complete problem is \link{https://en.wikipedia.org/wiki/Geometrical_frustration}{Geometrical Frustration}

\link{http://studenci.fuw.edu.pl/~kc427902/CMCS_2022/lab3.pdf}{Lecture Slides} \link{http://studenci.fuw.edu.pl/~kc427902/CMCS_2022/lect3.pdf}{Lab notes}
\end{lecture}

% -----------------------------------------------------------------------
% Wykład 24.03.2022

\begin{lecture}{Introduction to active matter}
    \emph{Active matter system types}\\
    We divide them into two types:
    \begin{itemize}
        \item Coloid systems
        \item Dry active matter systems
    \end{itemize}
     An example of dry active matter system is a one moving with \emph{Brownian Motion}. It is governed by \emph{The Langevin Equation}:
     \[
         m \ddot{x} = -\gamma \dot{x} + F(t)
     \]
     F(t) is a random force, we will assume it to be gaussian-distributed.
     The mean squared displacement for particles will be:
     \[
         \ev{(x(t) - x(0))^2} \sim \frac{\alpha}{\gamma^2} t
     \]
     Today we will be modelling a very viscous system, therefore negating the term $m \ddot{x}$.
\end{lecture}

\link{https://sites.google.com/uw.edu.pl/cmpp2022}{New Webpage}

For ex.2 Daniel recommends function \verb|scipy.spatial.KDTree|

\begin{lecture}{Tissue modelling as particles}

\end{lecture}

\printindex



\end{document}
